- Equations from Geophysics/Continuum Mechanics and High velocity impact physics that are needed to model the Impact

- all parameters that appear in the table of the Numerics section should be explained

\subsection{Conservation equations}
- mass
- momentum
- energy

\subsection{Constitutive equations}
- time evolution of deviatoric stress tensor


\subsection{Equation of State}
- Tillotson equation of state \cite{Tillotson_1962}

\subsection{Porosity model}
There are different ways in which porosity can be modeled depending on the pore size. Depending on the simulation, macro porosity with pore sizes above the resolution of the simulation can be accounted for in the initial conditions. This however becomes impossible for granular material with sub-resolution sized grains and pores.

Microporosity models porosity as an additional material property and can be applied independently of the resolution.


In these simulations, a microporosity p-$\alpha$ model as outlined in \cite{Jutzi_2008} is used. The distention $\alpha \in [1,\inf)$ relates the current density $\rho$ to the solid density $\rho_s$ which is reached if the material is fully compressed. For a non-porous material $\alpha$ equals one.

\begin{equation}
    \alpha \equiv \frac{\rho_s}{\rho}
\end{equation}

Often the porosity $\phi$ is used instead of the distention $\alpha$. They relate by

\begin{equation}
    \phi = 1 - \frac{1}{\alpha}
\end{equation}

- Quadratic crush curve

\subsection{Fragmentation model}
- fracture of brittle material
- Weibull distribution

\subsection{Strength model}
- elastic and plastic regimes
- von Mises strength
- pressure dependent yield strength

- ideas in \cite{Collins_2004}
- implementation in \cite{Jutzi_2015}