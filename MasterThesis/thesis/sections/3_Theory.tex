I will start with the fluid case for SPH and then show how it can be modified for solids.

\subsection{Incompressible Navier-Stokes Equations}
- quite new: developed in 1992
- goal was to solve Navier-Stokes Equations
- Fluide die Navier-Stokes Equations gehorchen heissen newtonsche fluide
- there are compressible and incompressible Navier Stokes Equations (for shockwaves)
- equations describe velocity changes (in a defined region of space) over time
- compressible N.S. Equations:
$\rho \cdot (\frac{\partial v}{\partial t} + v \cdot \nabla v) = \rho \cdot g - \nabla p + \mu \cdot \Delta v$
- convective acceleration (schneller bei Verengung) is a vector, not the result of a dot product:
$v\cdot \nabla v \equiv \begin{pmatrix} v_x \cdot \frac{\partial v_x}{\partial x} \\ v_y \cdot \frac{\partial v_y}{\partial y} \\ v_z \cdot \frac{\partial v_z}{\partial z}\end{pmatrix}$
- vector equation means three equations have to be solved independently
- Rearranged long vector form of navier stokes:
$ \rho \cdot \begin{pmatrix} \frac{\partial v_x}{\partial t} \\ \frac{\partial v_y}{\partial t} \\ \frac{\partial v_z}{\partial t}\end{pmatrix} =  - \rho \cdot \begin{pmatrix} v_x \cdot \frac{\partial v_x}{\partial x} \\ v_y \cdot \frac{\partial v_y}{\partial y} \\ v_z \cdot \frac{\partial v_z}{\partial z}\end{pmatrix} + \rho \cdot \begin{pmatrix} g_x \\ g_y \\ g_z \end{pmatrix} - \begin{pmatrix} \frac{\partial p}{\partial x} \\ \frac{\partial p}{\partial y} \\ \frac{\partial p}{\partial z}\end{pmatrix} + \mu \cdot \begin{pmatrix}  \frac{\partial^{2} v_x}{\partial x^{2}} + \frac{\partial^{2} v_x}{\partial y^{2}} + \frac{\partial^{2} v_x}{\partial z^{2}} \\ \frac{\partial^{2} v_y}{\partial x^{2}} + \frac{\partial^{2} v_y}{\partial y^{2}} + \frac{\partial^{2} v_y}{\partial z^{2}} \\ \frac{\partial^{2} v_z}{\partial x^{2}} + \frac{\partial^{2} v_z}{\partial y^{2}} + \frac{\partial^{2} v_z}{\partial z^{2}} \end{pmatrix} $
- laplacian on a vector field returns a vector, not a scalar!
- pressure (is this the same in miluphcuda?):
$ p = k \cdot (\rho - \rho_0) $
- Mass continuity (dividing by rho also gives velocity continuity!? - yes, but only in incompressible N.S. equations - what about solids?):
- $\rho \cdot (\nabla \cdot v) = 0 $
- how to solve N.S. and continuity at same time? ... automatically solved in particle methods when particles are not created/destroyed
- In SPH we want to know the local velocity change at each particle location
- Show that:
$ \frac{dv_i}{dt} = \frac{\partial v}{\partial t} + v \cdot \nabla v $

- $v \cdot \nabla v $ holds for any scalar field property y that a particle traverses with velocity u $u \cdot \nabla y $ since it is the projection $u \cdot \nabla y = |u|\cdot|\nabla y| \cdot cos (\Theta)$ where $\Theta$ is the angle between $u$ and $\nabla y$.
- Thus, the local change in the field property $\frac{dy_i}{dt}$ at the point i in the Lagrangian view corresponds to the sum of the change of the field property at a point i over time $\frac{\partial y_i}{\partial t}$ and the change $u_i \cdot \nabla y_i $ of the field property due to the movement of the particle along the field with velocity $u$ in the Eulerian view:
$ \frac{dy_i}{dt} = \frac{\partial y_i}{\partial t} + u_i \cdot \nabla y_i $

This expression is called the Material Derivative $\frac{Dy}{dt}$:
$ \frac{Dy}{dt} \equiv \frac{\partial y}{\partial t} + u \cdot \nabla y $

It gives us the change of the field property in the reference frame of the particle (denoted by the subscript i):

In the Navier-Stokes Equations the scalar fields of interest happen to be the individual velocity components so we get:
$\rho \cdot \frac{dv_i}{dt} = \rho \cdot \frac{Dv}{Dt} = \rho \cdot (\frac{\partial v}{\partial t} + v \cdot \nabla v) = \rho \cdot g - \nabla p + \mu \cdot \Delta v $

$ \frac{dv_i}{dt} = g - \frac{1}{\rho_i}\cdot \nabla p + \frac{\mu}{\rho_i} \cdot \Delta  v $
- This is where we change from Euler(grid methods) to Lagrangian view (particle information) ...

\subsection{Compressible Navier-Stokes Equations}
To apply the Navier-Stokes Equations to solids, a more general form is needed.



\subsection{Smoothed Particle Hydrodynamics}
Now that we have our compressible Navier-Stokes Equations in Lagrangian form we want to solve them.

- quite new: developed in 1992
- goal was to solve Navier-Stokes Equations
- Uses Smoothing Kernels ... similar concept as basis functions in FEM
- quantities are wighted average

- Terms in the Navier-Stokes Equation:
\begin{equation} \frac{dv_i}{dt} = g - \frac{1}{\rho_i}\cdot \nabla p + \frac{\mu}{\rho_i} \cdot \Delta  v \end{equation}

1) Density:
$ \rho_i \approx \sum_j m_j \cdot W(r - r_j,h) $

2)Pressure gradient:
$ \frac{\nabla p_i}{\rho_i} \approx \sum_j m_j \cdot (\frac{p_i}{\rho^2_i} + \frac{p_j}{\rho^2_j}) \cdot \nabla W(r - r_j, h) $
- why the gradient of W??

3) Viscosity term:
$ \frac{\mu}{\rho_i} \cdot \Delta v_i \approx \frac{\mu}{\rho_i} \cdot \sum_j m_j \cdot (\frac{v_j - v_i}{\rho_j}) \cdot \Delta W(r - r_j, h) $

Smoothing kernels:
- $ w = 0 for |r - r_0| > h $
$ \int_{||r' - r|| \leq h} W(r' - r, h)dr' = 1 $
- concrete Kernel implementation shown in Numerics section
